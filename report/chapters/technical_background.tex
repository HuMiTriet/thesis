\chapter{\centering Background}

Originally software architecture were monoliths where a single component is 
responsible for all the application's logic. However, over time especially how 
the issue of horizontally scaling the infrastructure become more and more pressing 
as the demand of the modern world for modern technology is also increasing for now
the solution to solve them is microservices.

This means that software systems are increasingly becoming distributed, but this 
also brings with it increasing challenges in ensuring the reliability and 
security of the system.

By introducing extra components into a software system also increases the amount 
of failure points that are possible. For example latency that are small in 
testing environment could become very substantial in the production environment 
consisting of hundreds of nodes.

This internship work aim is to show how much latency and error affects a distributed
system by doing fault injection, a technique where developers deliberately introduce
error into the system to see whether the system's response is sufficiently resilient 
or robust.

Furthermore, in the during the building of the prototype a library called Hypothesis
will be used for fuzzy based testing ensuring the correctness of the system. 

In addition, fuzzy based testing is also used to randomly inject faults into the system
to mimic the arbitrary nature of real world faults occurrences and avoiding covering
only the typically expected faults or "happy paths" \cite{happy_path}.
