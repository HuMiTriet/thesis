\chapter{\centering Background}

Traditionally, software architectures were monoliths, with a single component
responsible for all the application's logic. However, as the modern world's
demand for technology increased, the need for horizontal scaling of
infrastructure became more pressing. The solution has been a transition to
microservices.

As a result, software systems are becoming increasingly distributed. This
shift, however, brings challenges in ensuring the system's reliability and
security. The introduction of extra components into a software system increases
the potential failure points. For instance, latency that is minimal in a
testing environment can become substantial in a production environment with
hundreds of nodes.

The aim of this internship work is to explore how latency and errors impact a
distributed system through fault injection. This technique involves
deliberately introducing errors into the system to test its resilience and
robustness.

During the construction of the prototype, a library called Hypothesis was
employed for fuzz-based testing, ensuring the system's correctness.
Furthermore, fuzz-based testing was also used to randomly inject faults into
the system, mimicking the arbitrary nature of real-world fault occurrences.
This approach avoids focusing solely on the typically expected faults or "happy
paths" \cite{happy_path}.
