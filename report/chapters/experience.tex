\chapter{\centering Experience}

Prior to this, my experience was primarily in assisting with research processes
for professors and freelancing, particularly in web development. This formal
research role was a departure from those experiences, offering me a more
academic perspective.

I chose to work for the research group rather than taking a traditional
internship for two main reasons:

\begin{enumerate} \item I was interested in understanding what working in an
academic environment entails. \item I wanted to identify a specific field
within Computer Science that I could delve deeper into. \end{enumerate}

\textbf{Working in an Academic Environment}

The primary difference between my previous freelance work and this position was
the degree of freedom I enjoyed. Thanks to the guidance of my supervisor, Lukas
Atkinson, I had the opportunity to pursue various interests and ideas. However,
this freedom made it apparent that a balance between autonomy and guidance is
crucial. On several occasions, I implemented features that were unnecessary or
overly complex when simpler solutions would have sufficed.

One critical learning point was the importance of communication in experimental work.
Since neither party has a specific implementation in mind, it is essential to
remain aligned and focused on the broader goals.

The role required a high degree of independence, which made time management and
work-life balance crucial. There was a period at the beginning of the
internship when I overworked myself to the point of exhaustion. Atkinson's
advice helped me regain balance. We devised a concrete plan, set interim
deadlines, and prioritized tasks together. This made tasks more manageable and
less overwhelming.

This was one of the most important skills I learned during the internship. We
also set stretch goals - optional targets to aim for when progress was ahead of
schedule. This approach ensured a minimum set of goals were always met, and
additional tasks could be tackled if time allowed.

\textbf{Identifying My Field of Interest in Computer Science}

The internship made me realize that my knowledge base was still lacking.

Initially, I had a fixed mindset about what IT or cybersecurity entails.
However, the internship exposed me to the multifaceted nature of cybersecurity.
I understood that the primary goal is securing computational devices, and
limiting ourselves to a specific set of skills or knowledge can be
counterproductive. Therefore, our project explored various areas like fuzzy
testing, a bit of Machine Learning, and fault injection.

While I am still interested in cybersecurity, I now appreciate its diversity.
The field is composed of numerous subfields, each deploying or researching
different techniques and methods.
